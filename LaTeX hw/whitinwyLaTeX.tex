\documentclass[10pt, reqno]{article}

\usepackage{amsmath}

\title{\LaTeX \ Homework Assignment \\
Due 03/12/2021}
\author{Wyatt Whiting}
\date{March 12, 2021}

\begin{document}

\maketitle

Typeset the following lines (i.e. create an exact copy of this document using \LaTeX).

\section{First group of exercises}

\begin{enumerate}
\item $v = (v_1,v_2,v_3)^t$
\item $f_n(x) = x^n + a_{n - 1}x^{n - 1} + \cdots + a_1x + a_0$
\item $g(x)=\int_0^{\infty}G(t,x)dt$
\item $\sum_{i=1}^{\infty}x^i$
\item \[\sum_{i=1}^{\infty}x^i \]
\item We say that $\lambda$ is an eigenvalue of a matrix $A$ corresponding to eigenvector $u$ if $Au=\lambda u$. For example,

\[ 
\begin{bmatrix}
\ \  -1 & \frac{1}{3}\ \ \\
0 & 2 \ \ 
\end{bmatrix} 
\begin{bmatrix}
\ 1\ \\
\ 0\  
\end{bmatrix}
=
(-1)
\begin{bmatrix}
\ 1\  \\ 
\ 0\ 
\end{bmatrix}, 
\]


so $-1$ is an eigenvector of the matrix on the left corresponding to the eigenvector $(1, 0)^T$.

\end{enumerate}

\begin{center}
	\begin{tabular}{||l|r|}
	\hline
	\multicolumn{2}{||c|}{Fruit Prices} \tabularnewline
	\hline
	apples & \$.50 \tabularnewline
	peaches & \$1.25 \tabularnewline
	\hline
	\end{tabular}
\end{center}

\newpage

\section{Second group of exercises}

Here are some more nice exercises for \LaTeX. Try a piecewise defined function:

\[ 
f(x)=\begin{cases}
		x^2 \, &x > 0 \\
		0 \, &\text{otherwise} \\
	 \end{cases}
\]

Here is an example using Greek letters.

\[ 
\Lambda_N = \sum_{n=0}^N \frac{1}{\lambda^n}
\]



For the following, 	

\begin{align}
\sum_{i = 1}^{2^N - 1} \frac{1}{i} &= 1 + \frac{1}{2} + \cdots + \frac{1}{2^N - 1} \label{eq:1}\\
&> \frac{1}{2}+\frac{1}{4}+\frac{1}{4}+\overbrace{\frac{1}{8}+\cdots+\frac{1}{8}}^{\text{4 times}} + \cdots + \frac{1}{2^N}, \quad \text{for }N>3 \label{eq:2}
\end{align}

\noindent Note lines \eqref{eq:1} and \eqref{eq:2}. You must use \verb+\label{}+ and \verb+\eqref{}+ to make references to equations \eqref{eq:1} and \eqref{eq:2}.


\end{document}